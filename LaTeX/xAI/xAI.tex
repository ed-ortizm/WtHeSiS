\section{E\textbf{x}plainable \textbf{A}rtificial \textbf{I}ntelligence (xAI)}

\begin{enumerate}

  \item LIME

  \item SAHP
\end{enumerate}

\subsection{xAI for Unsupervised Learning}

LIME and SHAP are designed to interpret the predictions made by a given
classifier or regressor model $f$. Therefore, the $f$ comes from a SL setting.
For the case of the URF, this is not much of a problem because it is a
classifier that distinguishes whether a spectrum is real or synthetic. However,
the ODA based on the URF is a regressor since it outputs a continuous variable,
the outlier score. On the other hand, the AE is an UL algorithm, i.e, it is
neither a regressor nor a classifier. The ODA based on the AE has the workflow
illustrated in algorithm (\ref{alg:AE_ODA}), meaning this ODA can be framed as
a regressor. Having this in mind, we can conclude that for any ODA that computes
an outlier score, we can frame it as a regressor model $f$ for LIME and compute
an explanation model $\EModel$ for a given prediction.

\begin{algorithm}[tb]
\begin{algorithmic}
  \Require Observed spectrum $O$
  \Require Trained AE $AE$
  \Require Outlier score function $s$
  \State $R \gets AE.predict(O)$ \Comment{$R:\text{ reconstructed spectrum}$}
  \State $score \gets s(O,R)$  \Comment{$score:\text{outlier score of }O$}

  \Return $score$
\end{algorithmic}
\caption{ODA based on AE\label{alg:AE_ODA}}
\end{algorithm}

\import{./}{LIME.tex}
\import{./}{SHAP.tex}
